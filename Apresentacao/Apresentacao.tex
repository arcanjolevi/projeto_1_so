%% Incluindo os pacotes necessários
\documentclass{beamer}
\usepackage[utf8]{inputenc}
\usepackage[portuguese]{babel}

%\usetheme{Madrid}
%\usetheme{Copenhagen}
\usetheme{Ilmenau}
%\usecolortheme{beaver}
%\usecolortheme{orchid}
\usecolortheme{default}


%\usepackage{titlesec}
%\usepackage{titling}
%\usepackage{enumitem}
%\usepackage{indentfirst}
%\usepackage{graphicx}
%%\graphicspath{{images/}}
%\usepackage{fancyhdr}
%\usepackage{color}
%\usepackage{fancyhdr}
%\usepackage{colortbl}
%\usepackage{framed}

%% Definindo o Autor e o título
\author[Levi, Victor]{Levi Cícero Arcanjo  \and Victor Emanuel Almeida}
\title{Trabalho de SO}
%\newcommand{\prof}{Jorge Habib El Khouri}

\begin{document}
	\frame{\titlepage}
	\begin{frame}
		\frametitle{Conteúdo da apresentação}
		\tableofcontents
	\end{frame}
	\section{Introdução}

\begin{frame}{Algoritmos Utilizados}
    
	\begin{itemize}
		\item Round Robin: são atribuídas fatias iguais de tempo a cada processo, manipulando todos os processos sem prioridade e em ordem circular.
	\end{itemize}
\end{frame}

\begin{frame}{Algoritmos Utilizados}
    
	\begin{itemize}
		\item Shortest Job First: Algoritmo não preemptivo que seleciona para ser executado o processo com o menor tempo de execução.
	\end{itemize}
\end{frame}

\begin{frame}{Problema Clásico}
    
	\begin{itemize}
		\item Leitores e escritores
	\end{itemize}
\end{frame}


\begin{frame}
	teste
	
\end{frame}

	\begin{frame}
		texto p pagina \thepage\\
		\textbf{isso está em negrito}

		\begin{alertblock}{Coisa importante}
			Sample text in red box
		\end{alertblock}
	\end{frame}

	\begin{frame}
		\frametitle{teste}
		\begin{itemize}
			\item 
			\item 
			\item 
			\item 
			\item 
			\item 
		\end{itemize}
	\end{frame}
	\section{Ola mundo}
	\begin{frame}
		\frametitle{Ola mundo}
	\end{frame}
\end{document}